\documentclass[11pt,a4paper,english,final]{article}
\usepackage[english]{babel} % Using babel for hyphenation
\usepackage{lmodern} % Changing the font
\usepackage[utf8]{inputenc}
\usepackage[T1]{fontenc}

%\usepackage[moderate]{savetrees} % [subtle/moderate/extreme] really compact writing
\usepackage{tcolorbox}
\tcbuselibrary{hooks}
\usepackage[parfill]{parskip} % Removes indents
\usepackage{amsmath} % Environment, symbols etc...
\usepackage{amssymb}
\usepackage{float} % Fixing figure locations
\usepackage{multirow} % For nice tables
%\usepackage{wasysym} % Astrological symbols
\usepackage{graphicx} % For pictures etc...
\usepackage{enumitem} % Points/lists
\usepackage{physics} % Typesetting of mathematical physics examples: 
                     % \bra{}, \ket{}, expval{}
\usepackage{url}

\definecolor{red}{RGB}{255,10,10}

% To include code(-snippets) with æøå
\usepackage{listings}
\lstset{
language=c++,
showspaces=false,
showstringspaces=false,
frame=l,
}

\tolerance = 5000 % Bedre tekst
\hbadness = \tolerance
\pretolerance = 2000

\numberwithin{equation}{section}

\newcommand{\conj}[1]{#1^*}
\newcommand{\ve}[1]{\mathbf{#1}} % Vektorer i bold
\let\oldhat\hat
\renewcommand{\hat}[1]{\oldhat{#1}}
\newcommand{\trans}[1]{#1^\top}
\newcommand{\herm}[1]{#1^\dagger}

\newcommand{\Real}{\mathbb{R}}
\newcommand{\bigO}[1]{\mathcal{O}\left( #1 \right)}



\newcounter{algcounter}
\renewcommand{\thealgcounter}{\Roman{algcounter}}

\newenvironment{algorithm}{%
\refstepcounter{algcounter}
\begin{tcolorbox}
\centerline{Algorithm \thealgcounter}\vspace{2mm}
}
{\end{tcolorbox}}

\newcommand{\figurewidth}{.85\textwidth}

\title{FYS3150\\Computational Physics\\Project 3}
\author{Magnus Ulimoen}
\date{\today}

\begin{document}
\tcbset{before app=\parfillskip0pt}
\maketitle


\section{Introduction}

We are interested in the correlation energy between two electrons in
the helium atom. We do not have an analytical solution to this, so 
we assume that it is almost like two 1s-hydrogen atoms.


\section{Method}

\subsection{Wave function}

We overlay the simplest solution of the hydrogen atom on top of each
other, and end up with
\begin{gather}
\psi = e^{-\alpha r_1}e^{-\alpha r_2}
\end{gather}
Where we have discarded the normalization constant


\section{Implementation}
\subsection{Six-dimensional integral}
This is shortened to the loop
\begin{gather}
sum += \frac{f(r_1, r_2)}{\abs{\ve{r_1} - \ve{r_2}}} *w[i_x]\dots
\end{gather}


\end{document}
