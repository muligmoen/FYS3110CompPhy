\documentclass[11pt,a4paper,english,final]{article}
\usepackage[english]{babel} % Using babel for hyphenation
\usepackage{lmodern} % Changing the font
\usepackage[utf8]{inputenc}
\usepackage[T1]{fontenc}

%\usepackage[moderate]{savetrees} % [subtle/moderate/extreme] really compact writing
\usepackage{tcolorbox}
\tcbuselibrary{hooks}
\usepackage[parfill]{parskip} % Removes indents
\usepackage{amsmath} % Environment, symbols etc...
\usepackage{amssymb}
\usepackage{float} % Fixing figure locations
\usepackage{multirow} % For nice tables
%\usepackage{wasysym} % Astrological symbols
\usepackage{graphicx} % For pictures etc...
\usepackage{enumitem} % Points/lists
\usepackage{physics} % Typesetting of mathematical physics examples: 
                     % \bra{}, \ket{}, expval{}
\usepackage{url}

\definecolor{red}{RGB}{255,10,10}

% To include code(-snippets) with æøå
\usepackage{listings}
\lstset{
language=c++,
showspaces=false,
showstringspaces=false,
frame=l,
}

\tolerance = 5000 % Bedre tekst
\hbadness = \tolerance
\pretolerance = 2000

\numberwithin{equation}{section}

\newcommand{\conj}[1]{#1^*}
\newcommand{\ve}[1]{\mathbf{#1}} % Vektorer i bold
\let\oldhat\hat
\renewcommand{\hat}[1]{\oldhat{#1}}
\newcommand{\trans}[1]{#1^\top}
\newcommand{\herm}[1]{#1^\dagger}

\newcommand{\Real}{\mathbb{R}}
\newcommand{\bigO}[1]{\mathcal{O}\left( #1 \right)}

\newcommand{\di}{\mathrm{d}}

\newcounter{algcounter}
\renewcommand{\thealgcounter}{\Roman{algcounter}}

\newenvironment{algorithm}{%
\refstepcounter{algcounter}
\begin{tcolorbox}
\centerline{Algorithm \thealgcounter}\vspace{2mm}
}
{\end{tcolorbox}}

\newcommand{\figurewidth}{.85\textwidth}

\title{FYS3150 -- Computational Physics\\Project 3}
\author{Magnus Ulimoen}
\date{\today}

\begin{document}
\tcbset{before app=\parfillskip0pt}
\maketitle


\section{Introduction}

The goal of this project is to find the correlation energy between two 
electrons in the helium atom ground state. We do not have the analytical 
solution to this problem, so it is taken as an ansatz that the electrons 
can be superpositioned from hydrogen 1s-orbitals.

The program can be found under folder \url{Project3} in the github 
repository at \url{https://github.com/mulimoen/FYS3150CompPhy}.

\section{Method}

\subsection{Wave function}

The 1s-orbital for hydrogen has a wave function given by
\begin{gather}
\psi(\ve{r}) = R_n(r)Y_l^m(\theta, \phi)
\end{gather}
For hydrogen in 1s, the wave function does not depend upon angles,
\begin{gather}
\psi(r) = C e^{-\alpha \frac{r}{a_0}}
\end{gather}
Where C is a normalisation constant and $\alpha$ is a parameter 
determing the strength of the potential. 
For our helium-atom this is chosen $\alpha = Z = 2$, corresponding to 
a charge of $2e$ for the helium atom. $r$ will be used in natural units, 
where $a_0 = 1$ and the normalisation constant is discarded.

The wave function for the electrons can not be found exact, but an 
ansatz can be taken, that the wavefunctions is simply two overlapping 
hydrogen 1s-orbitals.
\begin{gather}
\psi(r_1, r_2) = e^{-\alpha(r_1 + r_2)}
\end{gather}


\subsection{Interaction energy}

Since the electrons are not localized, but rather spread out over all the 
space, it is necessary to find the expectation value of the energy. 
Classically the energy between two objects are given by
\begin{gather}
V = k\frac{qQ}{\abs{\ve{r}_1 - \ve{r}_2}} = k\frac{qQ}{\abs{\ve{r}_{12}}}
\end{gather}
This suggest that the quantum mechanical interaction energy is the 
ensemble over all the space,
\begin{gather}
E_{\text{interaction}} \propto  
\expval{\frac{1}{\abs{\ve{r}_1 - \ve{r}_2}}}
\end{gather}
Which can be written out 
\begin{gather}
E_{int} \propto \int\!\! \int \conj{\psi} 
\frac{1}{\abs{\ve{r}_{12}}} \psi 
\,\, \di\ve{r}_1\di\ve{r}_2 = I
\label{eq:I}
\end{gather}
An analytical solution of this is
\begin{gather}
I = \frac{5\pi^2}{16}
\end{gather}



\subsubsection{Cartesian coordinates}

The integral \eqref{eq:I} takes the following form in cartesian 
coordinates;
\begin{gather} I = 
\int\!\!\int\!\!\int\!\!\int\!\!\int\!\!\int
\frac{e^{-2\alpha\left(\abs{\ve{r}_1} + \abs{\ve{r}_2 }\right)}
}{\abs{\ve{r}_{12}}}
\di x_1\di y_1\di z_1\di x_2\di y_2\di z_2
\label{eq:Ecart}
\end{gather}
with all the limits from $-\infty$ to $\infty$


\subsubsection{Polar coordinates}
Since the integral constains a large amount of elements which depend 
on the radial distance, a change to polar coordinates could lead to 
better algorithms and solutions.

In order to change into polar coordinates we need to have the volume 
element $\di V$. This is given by 
\begin{gather}
\di V_1 = r_1^2\sin\theta_1 \di r_1\di\theta_1
\end{gather}
The term $\abs{\ve{r}_1 - \ve{r}_2}$ is expressed in these coordinates
\begin{gather}
\abs{\ve{r}_1 - \ve{r}_2} = \abs{r_1 \ve{e}_{r_1} - r_2\ve{e}_{r_2}}
 = \sqrt{(r_1 \ve{e}_{r_1} -  r_2\ve{e}_{r_2})\cdot 
 (r_1 \ve{e}_{r_1} - r_2\ve{e}_{r_2})}\\
 = \sqrt{r_1^2\ve{e}_{r_1}^2 + r_2^2\ve{e}_{r_2}^2  
 - 2r_1r_2 \ve{e}_{r_1}\cdot \ve{e}_{r_2}}
\end{gather}
The inner product of $\ve{e}_{r_i}$ with itself is one. It is necessary 
to find the dot product $\ve{e}_{r_1}\cdot \ve{e}_{r_2}$ which is the 
angle between the two unit vects, $\cos\beta$.
Writing out the radial unit vector in 
polar coordinates,
\begin{gather}
\ve{e}_r = \cos\phi\sin\theta \ve{i} + \sin\phi\sin\theta\ve{j} 
+ \cos\theta\ve{k}
\end{gather}
The inner product of the two radial unit vectors are 
\begin{gather}
\cos\beta = \cos(\theta_1)\cos(\theta_2)
+ \sin(\theta_1)\sin(\theta_2)\cos(\phi_1 - \phi_2)
\label{eq:cosbeta}
\end{gather}
And the integral \eqref{eq:I} takes the form
\begin{gather} I = 
\int\limits_0^{2\pi}\!\int\limits_0^{2\pi}\!
\int\limits_0^\pi\!\int\limits_0^\pi\!
\int\limits_0^\infty\!\int\limits_0^\infty 
\frac{r_1^2r_2^2 e^{-2\alpha(r_1 + r_2)} \sin(\theta_1)\sin(\theta_2)
}{\sqrt{ r_1^2 + r_2^2 - 2r_1r_2\cos\beta }}
\di r_1\di r_2\di\theta_1 \di\theta_2 \di\phi_1 \di\phi_2
\label{eq:IR}
\end{gather}

\section{Numerical approximations}

To solve this integral numerically some precautions must be taken.
A straight forward method requires all of the space to be integrated. 
The function should only be integrated where it is interresting, which 
is close to origo to due the decaying exponential. A suitable limit 
is chosen based on where the wave function is approximately zero.

Another problem is the singularity that arises when $\ve{r}_{12}$ is zero. 
This is solved by not letting this part contribute if this occurs.

The amount of variables used gives a large amount of intertwined 
variables that needs to be permuted. This gives a complexity of 
$N^6$ calculations of the function that is integrated. This 
requires better methods than linear mappings to solve, and Gauss 
Quadrature and Monte Carlo integration is employed.

\subsection{Gauss Quadrature}

The Gauss Quadrature uses orthogonal polynomials to capture the 
function in its space. The two methods used are Legendre polynomials 
and Laguerre polynomials.

\subsubsection{Legendre}

The Legendre 
quadrature uses the weight function 
\begin{gather}
I = \int_a^b f(y) dy = \int_{-1}^1 W(x) f(x) dx 
\approx \sum_i w_i f(x_i)
\end{gather}
The legendre polynomials are defined for integrals with limits $[-1,1]$.
A linear mapping is used, which maps the limits to $[0,\text{limit}]$.

\subsubsection{Laguerre}

This Laguerre quadrature uses the weight function
\begin{gather}
I = \int f(y) dy = \int_0^\infty \frac{W(x)}{x^\alpha e^{-x}} f(x) dx
\approx \sum_i w_i f(x_i)
\end{gather}
If this is used on \eqref{eq:IR}, 



\subsection{Monte Carlo}

In order to calculate the integral Monte Carlo methods are employed.

\subsubsection{Brute force}

We can map the cartesian integral from two limits, and then integrate 
over these using a linear mapping. We than map the interval $[0,1]$ to 
$[-\text{limit}, \text{limit}]$ to guess the solution.

The mapping is then $x_i = -\text{limit} + 
2\text{ limit } z_i$, where $z_i$ are the numbers 
produced by the random number generator. One would suspect this does 
not use the rng effective, as the function rapidly decays.

Normalisation factor is $(2\text{ limit})^6$, the size of the box 
which is integrated over.

\subsubsection{Radial MC}

Can use the radial coordinates instead, with different (linear) mapping,
$\theta_i = \pi z_i$, $\phi_i = 2\pi z_i$, $r_i = \text{limit} z_i$. 
We are then mapping the sphere to a box, and we can use the same method 
as above. The normalisation factor is $\pi^2(2\pi)^2(\text{limit})^2$,
which is the size of box.

\subsubsection{Importance sampling}

The method will not show good properties since the least important parts 
are sampled a lot more than necessary. It is therefore wise to have a 
closer look at the variables used and adapt the MC-sampling.

The spherical part has the factor $\sin\theta d\theta$. This can be 
changed to 
\begin{gather}
\frac{d\cos \theta}{d\theta} = \sin\theta\\
d\cos\theta = \sin \theta d\theta
\end{gather}
By replacing this with the uniform distribution the variable $d\cos\theta$
is replaced with dx, with the limit substitution $-1, 1$.

We can also simplify \eqref{eq:cosbeta} to use this choice of variable,
\begin{gather}
\cos\beta = x_1 x_2 + \sqrt{1-x_1^2}\sqrt{1-x_2^2}\cos(\phi_1-\phi_2)
\end{gather}

For the radial part we choose an exponential distribution. This gives 
\begin{gather}
\int_0^\infty p(y)f(y) dy = \int_0^\infty e^{-y}f(y) dy
\end{gather}
Our integral takes the form
\begin{gather}
\int \!\! dr \,\, r^2e^{-2\alpha r} f(r)\cdots\\
y = 2\alpha r = \frac{r}{\lambda}\\
r_i = -\lambda\ln(1-z)\\
\int_0^1 r^2f(r)
\end{gather}




\section{Implementation}
\subsection{Six-dimensional integral}
Looping over selected values of $x_1,x_2,y_1,y_2,\dots$ based on positions 
from GQ. This is a six-dimensional loop. In this loop there could be a 
singularity, which is ignored when the distance is to small.

Calculating the denominator of \eqref{eq:Ecart} and only calculating 
when this value is greater than some chosen tolerance. Then calculating 
r1 and r2, and getting the weights. This is added to the sum.

\subsubsection{Polar coordinates}
Much the same, but now r is used more, and 


\end{document}
