\documentclass[11pt,a4paper,final]{article}
\usepackage[english]{babel} % Using babel for hyphenation
\usepackage{lmodern} % Changing the font
\usepackage[utf8]{inputenc}
\usepackage[T1]{fontenc}


\usepackage[colorlinks=true]{hyperref}
\usepackage{cleveref}

%\usepackage[moderate]{savetrees} % [subtle/moderate/extreme] really compact writing
\usepackage{framed}
\usepackage{tcolorbox}
\tcbuselibrary{hooks}
\usepackage[parfill]{parskip} % Removes indents
\usepackage{amsmath} % Environment, symbols etc...
\usepackage{amssymb}
\usepackage{float} % Fixing figure locations
\usepackage{multirow} % For nice tables
%\usepackage{wasysym} % Astrological symbols
\usepackage{graphicx} % For pictures etc...
\usepackage{enumitem} % Points/lists
\usepackage{physics} % Typesetting of mathematical physics examples: 
                     % \bra{}, \ket{}, expval{}
\usepackage{url}


\usepackage{caption}
\usepackage{subcaption}

\definecolor{red}{RGB}{255,10,10}

% To include code(-snippets) with æøå
\usepackage{listings}
\lstset{
language=c++,
showspaces=false,
showstringspaces=false,
frame=l,
}

\tolerance = 5000 % Bedre tekst
\hbadness = \tolerance
\pretolerance = 2000

\numberwithin{equation}{section}

\newcommand{\conj}[1]{#1^*}
\newcommand{\ve}[1]{\mathbf{#1}} % Vektorer i bold
\let\oldhat\hat
\renewcommand{\hat}[1]{\oldhat{#1}}
\newcommand{\trans}[1]{#1^\top}
\newcommand{\herm}[1]{#1^\dagger}

\newcommand{\Real}{\mathbb{R}}
\newcommand{\bigO}[1]{\mathcal{O}\left( #1 \right)}

\newcommand{\di}{\mathrm{d}}
\newcommand{\magM}{\mathcal{M}}

\newcounter{algcounter}
\renewcommand{\thealgcounter}{\Roman{algcounter}}

\newenvironment{algorithm}{%
\refstepcounter{algcounter}
\begin{tcolorbox}
\centerline{Algorithm \thealgcounter}\vspace{2mm}
}
{\end{tcolorbox}}

\newcommand{\figurewidth}{.85\textwidth}

\title{FYS3150/4150\\Computational Physics\\Project 4}
\author{Magnus Ulimoen \& Krister Stræte Karlsen\\
Candidate numbers 33 \& 63}
\date{\today}

\begin{document}
\tcbset{before app=\parfillskip0pt}
\maketitle

\begin{abstract}
This project explores the Ising model applied to spins in two dimensions.
The critical temperature for the Ising model is determined to be 
$T_c =  2.27001284722$, which has a relative error $0.0004$ from the result 
of Lars Onsager. The energy, magnetisation, heat-capacity and 
magnetic susceptibility has been found for lattice sizes of 20x20, 40x40,
60x60, 80x80 and 100x100 near this temperature.
\end{abstract}



\section{Introduction}

The Ising model is a simple model of a solid. A lattice with elements 
each have a spin. This interaction between the different spin states
give an energy. The net amount of spin in the 
lattice leads to an overall magnetisation. 

Which energy the lattice 
is permitted to have is controlled by a large heat reservoir. If 
the temperature is large, the spins tends to be randomly aligned, 
and there is no net magnetisation. The highest temperature which 
allows a magnetisation is called the Curie temperature, and is 
connected to a phase transition.

The programs belonging to this report is available at 
\url{https://github.com/mulimoen/FYS3150CompPhy} under Project 4.

\section{Theory}

The energy of a spin system in the Ising model is given by
\begin{gather}
E = -J\sum_{\expval{kl}}^N s_k s_l
\end{gather}
Here J is the energy shared between two lattice points. The elements 
$s_i = \pm 1$ gives the spin associated with 
the lattice point.

The summation $\expval{kl}$ only sums over pairwise interactions.
Another way of writing the sum is to only consider the forward 
interaction between the lattice points,
\begin{gather}
E = -J \sum_{k}^N s_k s_{k+1}
\label{eq:Esum}
\end{gather}
This sum faces a boundary problem. In our model we are solving this 
by using periodic boundary conditions,
\begin{gather}
s_k = s_{k \pm L}
\label{eq:boundary}
\end{gather}
An illustration of this is visible in figure \ref{fig:spin_neighbours_full}.
Here the spin $S_2$ faces $S_4$ both on the upper and lower site.

\begin{figure}[H]
\centering
\includegraphics[scale=0.15]{pics/simple_lattice.png}
\caption{ Illustration of the simple, ''$L=2$''-lattice and a graphical interpretation the periodic boundary conditions.}
\label{fig:spin_neighbours_full}
\end{figure}

The magnetisation of the lattice is given by the sum of the spins,
\begin{gather}
\magM = \sum_i s_i
\label{eq:Msum}
\end{gather}


\subsection{Statistical mechanics}
In order to find how the energy and magnetisation is in the steady-state
solution some results from statistical mechanics is employed. These 
give expectation values in the thermodynamic limit, which is the limit 
where the number of samplings and particles goes to infinity.

The steady-state solution has a close connection to the temperature 
of the system. If our system is connected to a heat reservoir, we can
give the probability by the Boltzmann probability function,
where he probability to find the system in a microstate is given by
\begin{gather}
p(i) = \frac{e^{-E_i\beta}}{Z}
\end{gather}
Where $E_i$ is the energy for the microstate, $\beta = \frac{1}{kT}$ 
and Z is the partition function. The partition function works as the 
normalisation factor, and is given by
\begin{gather}
Z = \sum_i \exp(-\beta E_i)
\end{gather}

\subsubsection{Energy}
The energy of the system can be calculated from the partition function
by the equation
\begin{gather}
\expval{E} = -\frac{d \ln Z}{d \beta}
\end{gather}
One useful quantity is the heat capacity, how much energy the system 
retains when changing the temperature. This quantity is given by
\begin{gather}
C_v = \frac{d\expval{E}}{dT} 
= -\frac{\beta^2}{k}\frac{d\expval{E}}{d\beta}
 = \frac{\beta^2}{k}\frac{d^2\ln Z}{d^2\beta}
\end{gather}
Another way to write this
\begin{gather}
C_v = N\beta^2 \left(\expval{E^2} -\expval{E}^2\right)
\end{gather}
with $\expval{E}$ given per particle.




\subsubsection{Magnetisation}
From the partition function the magnetisation can be found. If all the 
states with a certain magnetisation is known, 
\begin{gather}
\expval{\magM} = \sum_i m_i p(m_i) = \sum_i \frac{M_i e^{-E_i\beta}}{Z}
\end{gather}
Another important quantity is the susceptibility. This is how much 
magnetisation is reatained when the temperature changes,
\begin{gather}
\chi = \frac{d \expval{\magM}}{d T} 
= -\frac{\beta^2}{k}\frac{d\expval{\magM}}{d\beta}
\end{gather}

\begin{gather}
\chi = N\beta^2 \left(\expval{M^2} - \expval{M}^2\right)
\end{gather}


\subsection{Example of a system}
For the simple case of only two spins in each dimension($L=2$)
the energies are summed by \eqref{eq:Esum}.
The energy a function of the different spins,
\begin{equation}
E = -2J(s_1s_2 + s_1s_3 + s_2s_4 + s_3s_4)
\end{equation}
This has symmetries with regards to which spin is up.
Listing up all the different spin configurations, their degeneracy 
and the energies,
\begin{gather*}
\text{Spin configuration} = 
\begin{cases}
n_{up} = 0, \quad d=1, \quad E = -8J\\
n_{up} = 1, \quad d=4, \quad E = 0\\
n_{up} = 2, \quad d=6, \quad 
\begin{cases} 
\text{diag} \quad d = 2, \quad E = 8J\\
\text{non-diag} \quad d = 4, \quad E = 0
\end{cases}\\
n_{up} = 3, \quad d=4, \quad E = 0\\
n_{up} = 4, \quad d=1, \quad E = -8J
\end{cases}
\end{gather*}
The energies with their degeneracy is then
\begin{gather}
E = 
\begin{cases}
-8J, \quad d = 2\\
\phantom{-}0, \quad\quad d = 12\\
\phantom{-}8J, \quad d = 2
\end{cases}
\end{gather}
And the (absolute) magnetisation is
\begin{gather}
\abs{\magM} = 
\begin{cases}
4, \quad d = 2,\quad E = -8J\\
2, \quad d = 8, \quad E = 0\\
0, \quad
\begin{cases}
d = 2, \quad E = 8J\\
d = 4, \quad E = 0
\end{cases}
\end{cases}
\end{gather}
The partition function for this system is 
\begin{gather}
Z = \sum_i d(i) e^{-E_i\beta} = 2e^{+8J\beta}+12e^{0}+2e^{-8J\beta}
 = 12 + 4\cosh(8J\beta)
\end{gather}


\subsubsection{Analytical solutions}
Using the partition function we obtain an expression for the energy;
\begin{gather}
\expval{E} = \frac{-32J\sinh(8J\beta)}{4\cosh(8J\beta)+12}
\end{gather}
magnetisation;
\begin{gather}
\expval{\abs{\magM}} = \frac{\abs{\magM_i}e^{-E_i\beta}}{Z}
= \frac{4e^{8J\beta} + 2}{12+4\cosh(8J\beta))}
\label{eq:absM}
\end{gather}
heat capacity;
\begin{gather}
C_v = -\frac{\beta^2}{k}\left(\frac{-256J^2\cosh(8J\beta)}{4\cosh(8J\beta) + 12}
 + \frac{1024J^2\sinh(8J\beta)^2}{(12+4\cosh(8J\beta))^2}\right)
\end{gather}
and magnetic susceptibility;
\begin{gather}
\chi = -\frac{\beta^2}{k}\left(- \frac{32 J \left(4 e^{8 J \beta} + 2\right) \sinh{\left (8 J \beta \right )}}{
\left(4 \cosh{\left (8 J \beta \right )} + 12\right)^{2}} 
+ \frac{32 J e^{8 J \beta}}{4 \cosh(8 J \beta) + 12}\right)
\end{gather}
The heat capacity and magnetic susceptibility follows from the derivative 
of $\expval{E}$ and $\expval{\abs{\magM}}$.

\subsection{1-D Ising model}
Solutions for the Ising model in one dimension in the 
thermodynamic limit are known, \cite{PB94}. 

\begin{equation}
\frac{U}{J} = -N \tanh \left( \frac{J}{k_B T} \right) = -N \frac{e^{J/k_B T}-e^{-J/k_B T} }{e^{J/k_B T}+ e^{-J/k_B T} } =  \begin{cases} N, \quad k_B T \to 0 \\ 0, \quad  k_B T \to \infty \end{cases}
\end{equation}

The specific heat per particle and the magnetisation are given by 

\begin{equation}
C(T) = \frac{1}{N} \frac{dU}{dT} = \frac{(J/k_B T)^2}{cosh^2 (J/k_B T)}
\end{equation}
\begin{equation}
\magM(T) = \frac{N e^{J / k_B T} sinh(B/k_B T) }{ \sqrt{ e^{2J/k_B T} sinh^2 (B/k_B T) + e^{-2J/ k_B T}  } }
\end{equation}

\subsection{2-D Ising model}
There are known analytical solution to the two dimensional model
as well \cite{KH87},
\begin{equation}
\magM (T) = \begin{cases} 0, \quad T > T_C \\ \frac{(1+z^2)^{1/4} (1-6z^2 + z^4 )^{1/8} }{(1-z^2)^{1/2}}, \quad T < T_C \end{cases}
\end{equation}
\begin{equation}
k T_C/J = \frac{2}{\ln(1+ \sqrt{2}) } \approx 2.269
\end{equation}

here $T_C$ is the \emph{Curie temperature} and $z = e^{-2J/k_B T}$. This result was originally derived by Lars Onsager.

\section{Implementation/Numerical}

The numerical model of the Ising model has been programmed in C++ and is 
available in the repository given in the introduction.
Most of the program code is well documented, and is available in the 
repository given above. The classes are tested using unit tests 
implemented in Catch, \url{https://github.com/philsquared/Catch}.

In the program a lattice class with periodic boundary conditions is given,
so access of an element follows the requirement in \eqref{eq:boundary}.
The energy of an element in the 
lattice and the total energy is calculated from the definition
of the energy, \eqref{eq:Esum}.

As a bonus a small program to plot the Ising model live is included. 
This allows one to see how the domains are connected to the temperature.
This program is available by running \emph{make animation}, and running 
the executable.


\subsection{The metropolis algorithm}

The probability of transition from i to j 
is modelled by the Boltzmann coefficient
\begin{gather}
w_j = \exp(-\beta (E_i - E_j))
\end{gather}
This suggest the following probabilities for moving to this state:
\begin{gather}
p = \begin{cases}
1 \quad\text{ if }  E_i - E_j \le 0\\
\exp(-\beta \Delta E) \text{ else}
\end{cases}
\label{eq:transition}
\end{gather}
This probability is compared with a random number generated from 
the computer. To avoid computing this exponential for each new state,
the total transition is modelled by flipping a single spin. The energy 
difference between the two microstates is computed by considering 
a a spin $S_0$ surrounded by $S_1,S_2,S_3, S_4$, as in figure 
\ref{fig:spin_neighbours}.

\begin{figure}[H]
\centering
\includegraphics[scale=0.2]{pics/full_lattice.png}
\caption{ Illustration of $S_0$ and its "nearest neighbours".}
\label{fig:spin_neighbours}
\end{figure}

This gives a flip of spin $s_0$ results in a n
energy-difference of the lattice which is
 -2 times the energy of $s_0$.
\begin{gather}
\Delta E = -2s_0(s_1 + s_2 + s_3 + s_4)
\end{gather}
This is clearly quantized, and the exponential function in 
\eqref{eq:transition} is precomputed to save computations.


This can be summarized in the following algorithm:
\begin{algorithm}
\begin{enumerate}[label=\arabic*)]
\item Choose start configuration. Common choices of the Ising model 
are random ($T= \infty$) and ordered ($T= 0$). 
\item Pick a random particle, $s_0$ by drawing two random integers,
$x$ and $y$ in the lattice domain.
\item Suggest a spin-flip and compute possible change in 
energy $\Delta E$.
\begin{align*} 
\Delta E = -2s_0(s_1+s_2+s_3+s_4), \quad \Delta E \in \{-8,-4,0,4,8 \} 
\end{align*} 
\item \textsc{If} $\Delta E \leq 0 \quad \to$ Accept spin-flip. \\
\textsc{Else} Draw random number, $r \in [0,1]$ \\
\textsc{	If} $r \leq exp(-\Delta E \beta ) \quad \to$ Accept spin-flip. \\
\textsc{Else} $\quad \to$ Reject flip. 

\item Repeat step 2) - 4) long enough for the system to be 
uncorrelated.

\item Collect data of interest. 

\item Repeat step 2) - 6). 
\end{enumerate}
\end{algorithm}

\subsubsection{Correlation time}
If the measurements are done as soon as a flip is tried, the system has 
not changed enough to represent a completely new independent 
measurement. It is therefore required that at least a certain amount 
of flips has been tried before sampling. This correlation time should 
at least be the amount of spins in the system. This is one Monte Carlo 
cycle. 


\subsubsection{Choice of random number generator}
The pseudo random number generator used for the simulations in 
this report is the Mersenne Twister engine
instantiated with the parameters of 
\emph{Mersenne Twister 19937 generator} seeded with the time.
This generator is stored in each model, which makes 
parallelisation trivial.

The Mersenne Twister is available in the \texttt{<random>} header 
of the standard
library in C++ and has a period of the \emph{Mersenne number},
$2^{(n-1)w}-1$.
The instantiation chosen gives $w=32$ and $n=624$, and a period which is 
far larger than the minimum requirement.

\subsection{Hot start and cold start}
The system can not be started in the most likely state, an as such two 
approximations are used. The first one sets all spins to be pointing in 
the same direction, which creates the lowest energy. This is the 
low temperature start, as the system is in the most likely state 
(ground state) for low temperatures. The other approach is to set 
all spins randomly, which is the high temperature approach, where 
all states are equally possible.

If a system is started with a random orientation it can have problems 
reaching the lowest state. There are areas with metastability, where 
there is a formation of magnetic domains, bands which crosses the lattice.
To break this up the temperature needs to be turned a little higher to 
increase the chance for the system to decrease in energy. The system 
needs to be thermally shaken.

An example of the formation of magnetic domains is visible in 
figure \ref{fig:mag_domain}, where the band is crossing the periodic 
boundary. All the simulations are therefore started with an ordered
set up.

\begin{figure}
\centering
\includegraphics[width=\figurewidth]{pics/mag_domain.png}
\caption{Example of a magnetic domain for $T = \frac{1}{10}$. Black 
and white are two different spin orientations}
\label{fig:mag_domain}
\end{figure}



\subsection{Computing thermodynamic properties}
The different thermodynamic properties of interest were computed 
according to the formulas \eqref{eq:Esum}, \eqref{eq:Msum} and
some modified formulas for the specific heat and magnetic susceptibility
are used. All these quantities are computed per particle. The 
sampling is repeated many times to obtain a reasonable mean.

The formulas for specific heat and magnetic susceptibility can be 
justified by physical means, where $\sigma$ gives 
the changes the small perturbations in temperature when a flip happens.
The difference will be smaller if the system is tightly bound, and it 
does not induce a big change in the energy or magnetisation.

\subsubsection{Specific heat}
\begin{equation}
C_v = \beta^2L^2\left(\expval{E^2} - \expval{E}^2\right)
\end{equation}

\subsubsection{Magnetic susceptibility}
\begin{equation}
\chi = \beta^2 L^2 \left(\expval{\magM^2} - \expval{\magM}^2\right)
\end{equation}



\subsection{Finite size scaling}
\label{subsec:Tc}

Near $T_C$ we can characterize the behaviour of many physical quantities
by a power law behaviour.
As an example the mean magnetization is given by
\begin{equation}
  \langle {\cal M}(T) \rangle \sim \left(T-T_C\right)^{\beta},
\end{equation}
where $\beta=1/8$ is a so-called critical exponent. A similar relation
applies to the heat capacity 
\begin{equation}
  C_V(T) \sim \left|T_C-T\right|^{\alpha},
\end{equation}
and the susceptibility
\begin{equation}
  \chi(T) \sim \left|T_C-T\right|^{\gamma},
\end{equation}
with $\alpha = 0$ and $\gamma = 7/4$.
Another important quantity is the correlation length, which is expected
to be of the order of the lattice spacing for $T>> T_C$. Because the spins
become more and more correlated as $T$ approaches $T_C$, the correlation
length increases as we get closer to the critical temperature. The divergent
behavior of $\xi$ near $T_C$ 
is
\begin{equation}
  \xi(T) \sim \left|T_C-T\right|^{-\nu}.
  \label{eq:xi}
\end{equation}
A second-order phase transition is characterized by a
correlation length which spans the whole system.
Since we are always limited to a finite lattice, $\xi$ will
be proportional with the size of the lattice. 
Through so-called finite size scaling relations
it is possible to relate the behaviour at finite lattices with the 
results for an infinitely large lattice.
The critical temperature scales then as
\begin{equation}
 T_C(L)-T_C(L=\infty) = aL^{-1/\nu},
 \label{eq:tc}
\end{equation}
for a constant, $a$ and  $\nu$ \cite{MHJ}. 


\subsection{Equilibration}
In order to find out how many steps are needed to to reach the most likely
state, we look at a 
$20 \cross 20$-lattice for different temperatures,
for both an ordered start configuration
and a random start configuration. 
This is done by an "by the eye"-approach, simply plotting the 
magnetisation and energy as a function of Monte Carlo cycles and see 
where they have only small fluctuations around some mean value. This is 
later used to as the starting point for gathering the data.

The flips needed before equilibrium has been reached is 
called \emph{thermalisation}. The number of accepted spin-flips as 
function of Monte Carlo cycles is also plotted to find the significance 
this has on the equilibrium.

\section{Results}

\subsection{Numerical and analytical results for 2x2-case}

In \cref{fig:aE,fig:aM,fig:acv,fig:achi}
 the analytical expressions for the 2x2-case is compared to the numerical.
The energy from the Ising model is closely following the analytical 
expression, especially for low temperatures. For higher temperatures
the energy tends to deviate from the analytical expression. The 
numerical magnetisation is following the expression for low 
temperatures, but does not fall off as rapidly as the analytical result.

The specific heat for the numerical model closely resembles the analytical
expression. The discrepancy between the analytical result and numerical 
expression is due to numerical artefacts. The magnetic susceptibility
is significantly off. The reason for this has not been discovered.


\begin{figure}
\centering
\includegraphics[width=\figurewidth]{pics/aE.png}
\caption{Analytical and numerical result for the energy 
of a 2x2 Ising model}
\label{fig:aE}
\end{figure}

\begin{figure}
\centering
\includegraphics[width=\figurewidth]{pics/aM.png}
\caption{Analytical and numerical result for the magnetisation 
of a 2x2 Ising model}
\label{fig:aM}
\end{figure}

\begin{figure}
\centering
\includegraphics[width=\figurewidth]{pics/acv.png}
\caption{Analytical and numerical result for the heat capacity 
of a 2x2 Ising model}
\label{fig:acv}
\end{figure}

\begin{figure}
\centering
\includegraphics[width=\figurewidth]{pics/achi.png}
\caption{Analytical and numerical result for the magnetic susceptibility
of a 2x2 Ising model}
\label{fig:achi}
\end{figure}




\subsection{Cold start and hot start}

The results for the different start configurations are visible 
in \cref{fig:EM_low_T,fig:EM_high_T}. For low 
temperatures the ordered system starts out in the most likely state. For 
the unordered system to enter the same state, it needs about 30000 total 
flips, or about 75 trial flips per site (Monte Carlo cycles) 
to enter this state.

If the acceptances are compared, it 
results in the figures \ref{fig:ca_low_T} and \ref{fig:ca_high_T}.



\begin{figure}
\centering
\includegraphics[width=\figurewidth]{pics/cEM1.png}
\caption{Energy and magnetisation as a function of Monte Carlo cycles
for $\tilde{T}$ = 1}
\label{fig:EM_low_T}
\end{figure}

\begin{figure}
\centering
\includegraphics[width=\figurewidth]{pics/cA1.png}
\caption{Accumulated acceptance 
for $\tilde{T}$= 1. MC cycles is the amount of flips tried}
\label{fig:ca_low_T}
\end{figure}

\begin{figure}
\centering
\includegraphics[width=\figurewidth]{{pics/cEM2.4}.png}
\caption{Energy and magnetisation as a function of Monte Carlo cycles
for $\tilde{T}$= 2.4}
\label{fig:EM_high_T}
\end{figure}

\begin{figure}
\centering
\includegraphics[width=\figurewidth]{{pics/cA2.4}.png}
\caption{Acumulated acceptance 
for $\tilde{T}$= 2.4. MC cycles is the amount of flips tried}
\label{fig:ca_high_T}
\end{figure}



\begin{figure}[H]
\centering
\includegraphics[scale=0.10]{pics/equilibrium_different_temperatures.png}
\caption{Equilibrium configuration for different 
temperatures, $\tilde{T}$. Here a $30 \times 30$ lattice for 
$\tilde{T} \in\{1.0, 2.0, 2.2, 2.3, 3.0\}$. One can see the behaviour of 
the system change close to the critical temperature where the phase 
transition happens. After the phase transition all magnetization is gone.} 
\label{fig:equilibrium_T}
\end{figure}

\subsection{Computing the thermodynamic properties}

The probabilities of the different energies are in figure 
\ref{fig:dT1} and \ref{fig:dT2.4}, where 50000 samplings
has been made.


\begin{figure}
\centering
\includegraphics[width=\figurewidth]{pics/dT=1.png}
\caption{Probabilities for the different energies for $\tilde{T}$ = 1}
\label{fig:dT1}
\end{figure}

\begin{figure}
\centering
\includegraphics[width=\figurewidth]{pics/{dT=2.4}.png}
\caption{Probabilities for the different energies for $\tilde{T}$ = 2.4}
\label{fig:dT2.4}
\end{figure}


\Cref{fig:e20,fig:e40,fig:e60,fig:e80,fig:e100,fig:e200} display the 
thermodynamic properties for changing lattice sizes.

\begin{figure}
\centering
\begin{subfigure}{0.49\textwidth}
\includegraphics[width=\textwidth]{pics/e20.png}
\caption{20x20 lattice}
\label{fig:e20}
\end{subfigure}
%
\begin{subfigure}{0.49\textwidth}
\centering
\includegraphics[width=\textwidth]{pics/e40.png}
\caption{40x40 lattice}
\label{fig:e40}
\end{subfigure}

\begin{subfigure}{0.49\textwidth}
\includegraphics[width=\textwidth]{pics/e60.png}
\caption{60x60 lattice}
\label{fig:e60}
\end{subfigure}
%
\begin{subfigure}{0.49\textwidth}
\centering
\includegraphics[width=\textwidth]{pics/e80.png}
\caption{80x80 lattice}
\label{fig:e80}
\end{subfigure}

\begin{subfigure}{0.49\textwidth}
\centering
\includegraphics[width=\textwidth]{pics/e100.png}
\caption{100x100 lattice}
\label{fig:e100}
\end{subfigure}
%
\begin{subfigure}{0.49\textwidth}
\centering
\includegraphics[width=\textwidth]{pics/e200.png}
\caption{200x200 lattice, $C_v$ is miscalculated due to overflow}
\label{fig:e200}
\end{subfigure}

\caption{Energy, magnetisations, specific heat and magnetic 
susceptibility for differently sized lattices}
\label{fig:EM}
\end{figure}

\subsubsection{The critical temperature and finite size scaling}
We will now see how close we can come to Lars Onsager's exact result 
of the critical temperature using the theory of finite size scaling from 
section \ref{subsec:Tc}.

From equation 3.10, using $\nu = 1$, having computed $T_C(L), L \in \{20,40,60,80 \}$ we have four equations and two unknowns. That means we can calculate sevral approximations for $a$, simply by subtracting one equation from the other. Obtaining the following expression for $a$:
\begin{equation}
a = \frac{T_C(L_j)-T_C(L_i)}{L_j - L_i}
\end{equation}

 We can do this for all the combinations of equation sets, and then 
 compute an average value of $a$. 
\begin{table}[!h]
\caption{ $T_C$ for different lattice sizes. The values for $T_C (L)$ 
where gathered by locating the peak in susceptibility and reading off 
the graphs.  }
\begin{center}
\vspace{2mm}
\begin{tabular}{| c | c | c |}
	\hline
	L  & $T_C(L)$  \\		\hline		
	$20$ & $2.34$ \\
    $40$ & $2.31$ \\
    $60$ & $2.32$ \\
    $80$ & $2.27$ \\
  \hline
\end{tabular}
\end{center}
\end{table} 

Finally, we can compute $T_C(L=\infty)$ using equation 3.10 and the numerical results above. Running the python script \texttt{finite\_size\_scaling.py} using the results from table 1 gives the following result:

\begin{framed}
Finite size scaling approximation of T\_c:  2.27001284722 \\
Relative error:  0.00044638484893
\end{framed}

We have come very close to Lars Onsager's original result of
$T_C = 2.269$. 


\subsection{Reproduction of results}

Benchmark calculations along with the programs used are available 
along with the program. These are available under the Reproduction folder.
This requires setting the variable
\emph{global\_seed} in \emph{main.cpp} to 42. More information is included
in the README in this folder. The seed for the figures in the report are 
set by the time of the runs.

\section{Discussion}

\subsection{Analytical and numerical 2x2}

The close relation between the analytical and numerical result in 
\cref{fig:aE,fig:acv} show a strong coupling between statistical 
physics and our model. Our system should therefore closely follow 
the ideal model proposed. The discrepancies in \cref{fig:aE,fig:achi}
shows that the magnetisation of the system differs significantly 
from the result from statistical physics. We believe this is due to an 
error in the expression \ref{eq:absM}, which gives the wrong analytical 
expression.


\subsection{Hot and cold start}
These show that the ordered and unordered state has the same flip-rate for 
high temperatures. The ordered start does start out with a slightly higher
flip-rate, although it quickly reaches the constant acceptance rate.

For low temperatures the random start has a very high
acceptance rate, which slows down as it reaches the most likely state.
Thereafter the ordered and random start shows the same flip-rate.

The flip-rate is thus a good indicator of when the system has reached the 
equilibrium phase. If a metastable situation is reached these two will 
not coincide, as a metastable state is more likely to flip than a stable
state.

\subsection{Thermodynamic properties}
For a temperature of 1, we get that
the lowest energy level is the most likely. The next energy level has a 
probability of 0.1, and the next one is even lower than this. 
Comparing with Boltzmann factor for a flip (which has energy 
$\Delta E = 8$) we see that the first state is over-represented.

A temperature of 2.4 gives a larger spread in the energy, with a 
peak around $E = -500$ and a large spread. This resembles a 
Gaussian distribution. With an increasing number of lattice sites 
we expect the results from statistical physics to align, and it should 
really be Gaussian.

If \cref{fig:e20,fig:e40,fig:e60,fig:e80,fig:e100,fig:e200}
thermodynamic properties go through drastic changes in the interval
$\tilde{T}\in (2.0,2.4)$. This is the trace of the phase transition.
The magnetisation goes to zero quickly as the energy increases. The 
cliff of the magnetisation gets steeper with the increasing temperature.

Both the specific heat and magnetic susceptibility changes rapidly 
around the critical temperature. This clearly show some kind of phase
transition, with $C_v$ and $\chi$ getting narrower and higher. The 
decrease towards the analytical $T_c$ gives the approximation to 
the critical temperature, which is in clear agreement with the 
analytical result from Onsager.


\subsection{Visualisation of Ising system}

With the help of the program to visualise the system, we can see the 
system by eye. For temperatures around 1 we can clearly see how the 
lattice gets ''devoured'' by one spin-type, or in some cases the 
magnetic domains (\ref{fig:mag_domain}) occurs. Temperatures well 
above the critical temperature looks completely random, with no 
resemblance of clumping. With temperatures around $T_c$ the clumping 
of domains happens, but they seem to fade away or float around the 
lattice. This can be seen as quite mesmerising, and is recommended 
if some therapy is required.

\section{Conclusion}

The Ising model has been implemented rather successfully, and the 
phase transition and critical temperature is discovered. The results 
between the results from statistical physics and the numerical 
integration used in here coincides, and we get very good results for 
the critical temperature, $T_c = 2.27001284722$, which only deviates 
from the result from statistical physics in the thermodynamic limit
with a relative error $0.0004$.



\begin{thebibliography}{9}
\bibitem{PB94}
Plischke, M. and Bergersen, B. 1994, \emph{Equilibrium statistical physics}, World Scientific Pub. Co., Singapore.


\bibitem{KH87}
Huang, K. 1987, \emph{Statistical mechanics}, Wiley, New York. 


\bibitem{MHJ}
Hjort-Jensen,M., 2015. \emph{Project 4}, accessible at Github: 
\url{https://github.com/mhjensen} .

\end{thebibliography}


\end{document}