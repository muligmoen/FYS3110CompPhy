\documentclass[11pt,a4paper,english,draft]{article}
\usepackage[english]{babel} % Using babel for hyphenation
\usepackage{lmodern} % Changing the font
\usepackage[utf8]{inputenc}
\usepackage[T1]{fontenc}

%\usepackage[moderate]{savetrees} % [subtle/moderate/extreme] really compact writing
\usepackage{tcolorbox}
\tcbuselibrary{hooks}
\usepackage[parfill]{parskip} % Removes indents
\usepackage{amsmath} % Environment, symbols etc...
\usepackage{amssymb}
\usepackage{float} % Fixing figure locations
\usepackage{multirow} % For nice tables
%\usepackage{wasysym} % Astrological symbols
\usepackage{graphicx} % For pictures etc...
\usepackage{enumitem} % Points/lists
\usepackage{physics} % Typesetting of mathematical physics examples: 
                     % \bra{}, \ket{}, expval{}
\usepackage{url}

\definecolor{red}{RGB}{255,10,10}

% To include code(-snippets) with æøå
\usepackage{listings}
\lstset{
language=c++,
showspaces=false,
showstringspaces=false,
frame=l,
}

\tolerance = 5000 % Bedre tekst
\hbadness = \tolerance
\pretolerance = 2000

\numberwithin{equation}{section}

\newcommand{\conj}[1]{#1^*}
\newcommand{\ve}[1]{\mathbf{#1}} % Vektorer i bold
\let\oldhat\hat
\renewcommand{\hat}[1]{\oldhat{#1}}
\newcommand{\trans}[1]{#1^\top}
\newcommand{\herm}[1]{#1^\dagger}
%\renewcommand{\thefootnote}{\fnsymbol{footnote}} % Gir fotnote-symboler
\newcommand{\Real}{\mathbb{R}}
\newcommand{\bigO}[1]{\mathcal{O}\left( #1 \right)}

\newcommand{\spac}{\hspace{5mm}}

\newcounter{algcounter}
\newcommand{\algnum}{\stepcounter{algcounter}\Roman{algcounter}}

\title{FYS3150/4150\\Computational Physics\\Project 2}
\author{Magnus Ulimoen\\Krister Stræte Karlsen}
\date{\today}

\begin{document}
\tcbset{before app=\parfillskip0pt}
\maketitle

\section{Introduction}

In this projct the Schrödinger's equation for two electrons in a
three-dimensional harmonic oscillator well will be solved numerically. 
Both computational and physical aspects of the problem will be discussed 
throughout the report. 


\section{Method}

\subsection{Schrödinger's equation}


\subsubsection{Time-independent Schrödinger's equation}
From quantum mechanics we have Schrödinger's equation on the form
\begin{equation}
\hat{H} \ket{\Psi} = i\hbar \frac{\partial }{\partial t} \ket{\Psi}
\label{eq:Schrödinger}
\end{equation}
If the Hamiltonian $\hat{H}$ is independent of time this is separable,
and the left side of \eqref{eq:Schrödinger} is taken as E, and we get the
time-independent Schrödinger equation;
\begin{equation}
 \hat{H}\ket{\Psi} = E\ket{\Psi}
\end{equation}
For a single electron the Hamiltonian has the form 
\begin{equation}
\hat{H} = \frac{\hat{p}^2}{2m} + V(\ve{r})
\end{equation}
Or if written in the function basis,
\begin{equation}
\hat{H} = -\frac{\hbar^2}{2m}\nabla^2 + V(\ve{r})
\label{eq:Hamiltonian}
\end{equation}



\subsubsection{Spherical harmonics}
Writing out \eqref{eq:Hamiltonian} in spherical coordinates we end up
with the equation
\begin{gather}
\nonumber \hat{H} = \\
-\frac{\hbar^2}{2m}\left(\frac{1}{r^2}\frac{\partial}{\partial r}
\left( r^2\frac{\partial }{\partial r}\right)
+ \frac{1}{r^2}\left[ \frac{1}{\sin\theta}\frac{\partial}{\partial \theta}
\left( \sin\theta \frac{\partial}{\partial \theta}\right)
+ \frac{1}{\sin^2\theta}\frac{\partial^2}{\partial \phi^2}
\right]\right) + V(r,\theta,\phi)
\end{gather}
If $V$ is spherically symmetrical, the solution is separable into
$\Psi = R(r)Y_l^m(\theta, \phi)$ where $Y_l^m$ is the associated Laguerre
polynomials. The part in the square brackets is an eigenfunction of 
$Y_l^m$, with eigenvalue $-l(l+1)$. 


\subsubsection{Radial equation}

The radial part of the Schrödinger's equation now reads:
\begin{gather}
  -\frac{\hbar^2}{2 m} \left ( \frac{1}{r^2} \frac{d}{dr} r^2
  \frac{d}{dr} - \frac{l (l + 1)}{r^2} \right )R(r) 
     + V(r) R(r) = E R(r)
\label{eq:radial}
\end{gather}
The $l$ quantum number tends to ''throw'' the radial part outwards,
and works as a sentrifugal barrier.

\subsubsection{Harmonic oscillator}
In our case $V(r)$ is the harmonic oscillator potential 
$V = \frac{1}{2}k r^2$ with
$k=m\omega^2$. 

We make the substitution $R(r) = (1/r) u(r)$ in the radial
equation and obtain
\begin{gather}
  -\frac{\hbar^2}{2 m} \frac{d^2}{dr^2} u(r) 
       + \left ( V(r) + \frac{l (l + 1)}{r^2}\frac{\hbar^2}{2 m}
                                    \right ) u(r)  = E u(r) .
\end{gather}
The boundary conditions are $u(0)=0$ and $u(\infty)=0$ to avoid
an unnormalizable wavefunction.

Introducing the dimensionless variable $\rho = (1/\alpha) r$
where $\alpha$ is a constant with dimension length we get
\begin{gather}
  -\frac{\hbar^2}{2 m \alpha^2} \frac{d^2}{d\rho^2} u(\rho) 
       + \left ( V(\rho) + \frac{l (l + 1)}{\rho^2}
         \frac{\hbar^2}{2 m\alpha^2} \right ) u(\rho)  = E u(\rho) .
\end{gather}
In this project we will only study the situation $l=0$.
Inserting $V(\rho) = (1/2) k \alpha^2\rho^2$ we end up with
\begin{gather}
  -\frac{\hbar^2}{2 m \alpha^2} \frac{d^2}{d\rho^2} u(\rho) 
       + \frac{k}{2} \alpha^2\rho^2u(\rho)  = E u(\rho) .
\end{gather}
Multiplying with $2m\alpha^2/\hbar^2$ on both sides to obtain
\begin{gather}
  -\frac{d^2}{d\rho^2} u(\rho) 
       + \frac{mk}{\hbar^2} \alpha^4\rho^2u(\rho) 
       = \frac{2m\alpha^2}{\hbar^2}E u(\rho) .
\end{gather}
The constant $\alpha$ can now be fixed such that
\begin{gather}
\frac{mk}{\hbar^2} \alpha^4 = 1,
\end{gather}
and the constant is now
\begin{gather}
\alpha = \left(\frac{\hbar^2}{mk}\right)^{1/4}
\end{gather}
Defining 
\begin{gather}
\lambda = \frac{2m\alpha^2}{\hbar^2}E,
\end{gather}
we can rewrite Schrödinger's equation as
\begin{equation}
  -\frac{d^2}{d\rho^2} u(\rho) + \rho^2u(\rho)  = \lambda u(\rho) .
\end{equation}

This equation can now be discretized and solved as a linear system 
using for example Jacobi's algorithm for eigenvalues. 

\subsubsection{Eigenvalues of the energy}
The eigenvalues of \eqref{eq:radial} can be found analytically,
and are the energies
\begin{gather}
E_{nl}=  \hbar \omega \left(2n+l+\frac{3}{2}\right),
\end{gather}
where $n$ and $l$ are quantum numbers, with $n = 0, 1, 2, \dots$ and
$l = 0, 1, \dots, n-1$.
 
 
\subsection{Discretization}

Using a standard center approximation for the second derivative:$u$
\begin{equation}
    u''=\frac{u(\rho+h) -2u(\rho) +u(\rho-h)}{h^2} +O(h^2),
    \label{eq:diffoperation}
\end{equation} 
where $h$ is our steplength.

For a finite value of $\rho_{\mathrm{max}}$, we can define 
\begin{gather}
  h=\frac{\rho_{\mathrm{max}}-\rho_{\mathrm{min}} }{n_{\mathrm{step}}}.
\end{gather}
and 
\begin{gather}
    \rho_i= \rho_{\mathrm{min}} + ih, \hspace{1cm} i=0,1,2,\dots ,
    n_{\mathrm{step}}
\end{gather}
Then rewrite the Schrödinger equation for $\rho_i$ as
\begin{gather}
-\frac{u(\rho_i+h) -2u(\rho_i) +u(\rho_i-h)}{h^2}+\rho_i^2u(\rho_i) 
= \lambda u(\rho_i)
\end{gather}
where $V_i=\rho_i^2$ is the harmonic oscillator potential.

Defning 
\begin{gather}
   d_i=\frac{2}{h^2}+V_i, \hspace{0.5cm } e_i=-\frac{1}{h^2}.
\end{gather}
the Schrödinger equation takes the following form
\begin{gather}
d_iu_i+e_{i-1}u_{i-1}+e_{i+1}u_{i+1}  = \lambda u_i,
\end{gather}
where $u_i$ is unknown. We can write the 
latter equation as a matrix eigenvalue problem 
\begin{equation}
\begin{pmatrix} d_1 & e_1 & 0   & 0    & \dots  &0     & 0 \\
                e_1 & d_2 & e_2 & 0    & \dots  &0     &0 \\
                0   & e_2 & d_3 & e_3  &0       &\dots & 0\\
        \dots  & \dots & \dots & \ddots  &\ddots      &\dots & \dots\\
 0   & \dots & \dots & \dots  &\dots  &d_{n_{\mathrm{step}}-2} & e_{n_{\mathrm{step}}-1}\\
 0   & \dots & \dots & \dots  &\dots       &e_{n_{\mathrm{step}}-1} & d_{n_{\mathrm{step}}-1}
             \end{pmatrix}
\begin{pmatrix} u_{1} \\
      u_{2} \\
      \dots\\ \dots\\ \dots\\
      u_{n_{\mathrm{step}}-1}
\end{pmatrix} 
= \lambda \begin{pmatrix} u_{1} \\
                          u_{2} \\
                          \dots\\ \dots\\ \dots\\
                          u_{n_{\mathrm{step}}-1}
             \end{pmatrix} 
      \label{eq:sematrix}
\end{equation} 

We will now move on the see how we can solve this eivenvalue problem using Jacobi's algorithm. 
 
\subsection{Jacobi's eigenvalue algorithm}

\textcolor{red}{KSK: These are straight up answers to a). A more complete subsection should be made.}

Having defined the quantities $\tan\theta = t= s/c$, with $s=\sin\theta$ and $c=\cos\theta$ and
\begin{gather}\cot 2\theta=\tau = \frac{a_{ll}-a_{kk}}{2a_{kl}}.
\end{gather}

We can then define the angle $\theta$ so that the non-diagonal matrix elements of the transformed matrix 
$a_{kl}$ become non-zero and
we obtain the quadratic equation (using $\cot 2\theta=1/2(\cot \theta-\tan\theta)$
\begin{gather}
t^2+2\tau t-1= 0,
\end{gather}
resulting in 
\begin{gather}
  t = -\tau \pm \sqrt{1+\tau^2},
\end{gather}

Here a clever choice of roots must be made. For for a well trained computational physisist this equation should be a red warning sign of loss of numerical precision. To avoid subtraction of to almost equal numbers the following chioice should be made:

\begin{align*}
t=  -\tau + \sqrt{1+\tau^2}, \quad   0  \geq \tau \\
t=  -\tau - \sqrt{1+\tau^2}, \quad   \tau <  0
\end{align*}

From the equations above we obtain the folowing relation for $c$ and $t$
\begin{gather}
   c = \frac{1}{\sqrt{1+t^2}}.
\end{gather}
Knowing that $t \in [0,1]$ we put restrictions such that $c \in [\frac{1}{\sqrt{2}}, 1]$ which focres $\theta$ to be in the interval $[- \frac{\pi}{4},\frac{\pi}{4}]$.  

Chosing the smaller $t$ makes $c$ the larger and minimizes 
\begin{gather}
||{\bf B}-{\bf A}||_F^2=4(1-c)\sum_{i=1,i\ne k,l}^n(a_{ik}^2+a_{il}^2) +\frac{2a_{kl}^2}{c^2}.
\end{gather}


\subsubsection{Implementation}

\centerline{Algorithm \algnum}
\begin{tcolorbox}
Find the largest element, $a_{k,l}$, in $A$ \\
Compute $\tau:$ \\
$ \tau = \frac{a_{ll}-a_{kk}}{2a_{kl}}$ \\
Compute t according to:  \\
$t=  -\tau + \sqrt{1+\tau^2}, \quad   0  \geq \tau \\
t=  -\tau - \sqrt{1+\tau^2}, \quad   \tau <  0 $ \\
Set \\
$ c= \frac{1}{\sqrt{1+t^2}}, \quad s= ct$ \\
Compute the new matrix elements: \\
$ b_{kl} = a_{kl}, \quad k,l \neq i,j$ \\
$ b_{ik} = b_{ki} = ca_{ik}-sa_{jk}, \quad k \neq i,j$ \\
$ b_{jk} = b_{kj} = sa_{ik}+ca_{jk}, \quad k \neq i,j$ \\
$ b_{ij} = b_{ji} = (c^2 - s^2)a_{ij} + cs(a_{ii}-a_{jj}) $ \\
$ b_{ii} = c^2 a_{ii} - 2csa_{ij} + s^2 a_{jj} $ \\
$ b_{jj} = s^2 a_{ii} + 2csa_{ij} + c^2 a_{jj} $ 
\end{tcolorbox} 

\subsubsection{Unit tests}


\section{Results and reflection}

\section{Concluding remarks }

\end{document}
