\documentclass[11pt,a4paper,english]{article}
\usepackage[english]{babel} % Using babel for norwegian hyphenation
\usepackage{lmodern} % Changing the font
\usepackage[utf8]{inputenc}
\usepackage[T1]{fontenc}

%\usepackage[moderate]{savetrees} % [subtle/moderate/extreme] really compact writing
\usepackage{tcolorbox}
\usepackage[parfill]{parskip} % Removes indents
\usepackage{amsmath} % Environment, symbols etc...
\usepackage{amssymb}
%\usepackage{wasysym} % Astrological symbols
\usepackage{graphicx} % For pictures etc...
\usepackage{enumitem} % Points/lists
\usepackage{physics} % Typesetting of mathematical physics examples: \bra{}, \ket{}, expval{}

% To include code(-snippets) with æøå
%\usepackage{listings}
%\lstset{
%language=python,
%showspaces=false,
%showstringspaces=false,
%frame=l,
%literate=%
%{æ}{{\ae}}1
%{ø}{{\o}}1
%{å}{{\aa}}1
%{Æ}{{\AE}}1
%{Ø}{{\O}}1
%{Å}{{\AA}}1
%}

\tolerance = 5000 % Bedre tekst
\hbadness = \tolerance
\pretolerance = 2000

\newcommand{\conj}[1]{#1^*}
\newcommand{\ve}[1]{\mathbf{#1}} % Vektorer i bold
\let\oldhat\hat
\renewcommand{\hat}[1]{\mathbf{\oldhat{#1}}}
\newcommand{\trans}[1]{#1^\top}
\newcommand{\herm}[1]{#1^\dagger}
%\renewcommand{\thefootnote}{\fnsymbol{footnote}} % Gir fotnote-symboler
\newcommand{\Real}{\mathbb{R}}


\newcommand{\spac}{\hspace{5mm}}

\title{FYS3150/FYS4150 \\Project 1}
\author{Magnus Ulimoen\\Krister Stræte Karlsen}
\date{\today}

\begin{document}
\maketitle

\section{Motivation}

Solutions of the Poissons equation are a vital component to many aspects physics.


\section{a)}

%Moving $h^2$ over and setting
%\begin{gather*}
%A = \begin{pmatrix} 2 & -1 & \ldots & \ldots & 0\\
%		    -1 & 2 & 0 & \ldots & \ldots \\
%		    \end{pmatrix}
%\end{gather*}
\subsection*{Formulation}

Given the following ODE with boundary conditions, 

\begin{equation}
-u''(x) = f(x), \spac x \in (0,1), \spac u(0) = u(1) = 0.
\end{equation}

and the dicretized form following a symmetric Taylor expansion

\begin{equation}
-\frac{v_{i+1}+v_{i-1}-2v_i}{h^2} = f_i  \spac \mathrm{for} \spac i=1,\dots, n, \spac v_0 = v_{n} = 0
\label{eq:discretized}
\end{equation}

we are going to show it can be written as system of linear equations of the form: 

\begin{equation}
   \ve{A}\ve{v} = \tilde{\ve{b}},
   \label{eq:Avb}
\end{equation}


\subsection*{Solution}
Multipling the discretized equation \eqref{eq:discretized} by $h^2$ we get:

\begin{gather*}
   -v_{i-1}+2v_{i}-v_{i+1}=h^2 f_i \hspace{0.5cm} \mathrm{for} \hspace{0.5cm} i=1,\dots, n 
\end{gather*}

Filling in for $i$ and choosing $\tilde{b_i} = h^2 f_i$ we obtain the following set of equations: 
\begin{align*}
	2v_{1}-v_{2}=\tilde{b_1} \\
	-v_{1}+2v_{2}-v_{3}=\tilde{b_2} \\
	\vdots \hspace{0.5cm} \\
	-v_{i-1}+2v_{i}-v_{i+1}=\tilde{b_i} \\
	\vdots \hspace{0.5cm}  \\ 
	-v_{n-1}+2v_{n}=\tilde{b_n} \\
\end{align*}
Now one can easily see that this system of linear equations can written on the form of \eqref{eq:Avb},
where 
\begin{gather*}
    \ve{A} = \begin{pmatrix}%{cccccc}
                           2& -1& 0 &\dots   & \dots &0 \\
                           -1 & 2 & -1 &0 &\dots &\dots \\
                           0&-1 &2 & -1 & 0 & \dots \\
                           & \dots   & \dots &\dots   &\dots & \dots \\
                           0&\dots   &  &-1 &2& -1 \\
                           0&\dots    &  & 0  &-1 & 2 \\
                      \end{pmatrix},\spac \ve{v} = \begin{pmatrix}
                           v_1\\
                           v_2\\
                           \dots \\
                          \dots  \\
                          \dots \\
                           v_n\\
                      \end{pmatrix},
  \spac \tilde{\ve{b}} = \begin{pmatrix}
                           \tilde{b}_1\\
                           \tilde{b}_2\\
                           \dots \\
                           \dots \\
                          \dots \\
                           \tilde{b}_n\\
                      \end{pmatrix}.
\end{gather*}


\newpage

\subsection*{Algorithm for a tridiagonal system of linear equations}

We start by looking at the system of equations:
\begin{align*}
	 b_1 v_1 + c_1 v_{2} = \tilde{b_1} \hspace{0.5cm} (1*)\\
	 a_2 v_1 + b_2 v_2 + c_3 v_3 = \tilde{b_2} \hspace{0.5cm} (2*) \\
	 a_3 v_2 + b_3 v_3 + c_3 v_4 = \tilde{b_3} \hspace{0.5cm} (3*) \\
	 \vdots \hspace{0.5cm}  \\ 
	 a_n v_{n-1} + b_n v_n = \tilde{b_n} \hspace{0.5cm} (n*)\\
\end{align*}
If we solve (1*) for $v_1$ and insert it into (2*) we obtain the following "modified second equation":
\begin{align*}
	(b_1 b_2 - a_2 c_1)v_2 + b_1 c_2 v_3 = b_1 \tilde{b_2} - a_2  \tilde{b_1}  
\end{align*}
Now having successfully removed $v_1$ from the second equation we can go on and solve it for $v_2$ and insert it into the third equation obtaining:
\begin{align*}
(b_3 (b_1 b_2 - a_2 c_1)- a_3 b_1 c_2)v_3 + c_3(b_1 b_2 -a_2 c_1)v_4 = (b_1 b_2 - a_2 c_1 ) \tilde{b_3} - a_3 b_1 \tilde{b_2} + a_2 a_3 \tilde{b_1} 
\end{align*}

The two modified equations may be written as 

\begin{align*}
v_2 = \frac{b_1 \tilde{b_2} - a_2 \tilde{b_1}}{b_1 b_2 - a_2 c_1} - \frac{b_1 c_2}{b_1 b_2 - a_2 c_1} v_3 = \beta_3 + \gamma_3 v_3
\end{align*}
\begin{align*}
v_3 =& \frac{(b_1 b_2 - a_2 c_1) \tilde{b_3} - a_3 (b_1 \tilde{b_2} - a_2 \tilde{b_1})}{b_3 (b_1 b_2 - a_2 c_1)- a_3 b_1 c_2} - \frac{c_3(b_1 b_2 - a_2 c_1)}{b_3 (b_1 b_2 - a_2 c_1)- a_3 b_1 c_2} v_4 \\
=& \beta_4 + \gamma_4 v_4 = \frac{\tilde{b_3}-a_3 \beta_3}{a_3 \gamma_3 + b_3} + \frac{-c_3}{a_3 \gamma_3 + b3}v_4
\end{align*}

This prossess can be repeated up untill the last equation. This is the forward substitution step. From the last equation we compute $v_n$ and get all we need to compute $v_{n-1}$, then $v_{n-2}$, and so on. This is the backward substitution part of the algorithm. A shrewd reader might see that the coefficiants, $\beta$ and $\gamma$, take a recursive form
\begin{align*}
\beta_{i+1} = \frac{\tilde{b_i}-a_i \beta_i}{a_i \gamma_i + b_i} , \hspace{0.5cm} \gamma_{i+1}=\frac{-c_i}{a_i \gamma_i + b_i},
\end{align*}
and the equation for $v_{i-1}$ reads: 
\begin{equation}
v_{i-1} = \beta_i + \gamma_i v_i
\end{equation}

It follows from (1*) that $\beta_1 = \gamma_1 = 0$.
From combining (n*) and (4) we get 
\begin{align*}
v_n = \frac{\tilde{b_n}-a_n \beta_n}{a_n \gamma_n + b_n} = \beta_{n+1} + \gamma_{n+1} v_{n+1}, \hspace{0.5cm} v_{n+1} = 0.
\end{align*}

The goal of the seemingly stupid formulation above is to underline the importance of setting $v_{n+1}=0$ in order to obtain the right formula for $v_n$.

Having all the nessecary ingredients the algorithm reads as follows.
\vspace{0.5cm}\\
\centerline{Algorithm I}
\begin{tcolorbox}
$a_i = c_i = -1, \hspace{0.5cm}  i=1,2,3,..,n$ \\
$b_i = 2, \hspace{0.5cm}  i=1,2,3,..,n $\\
$\tilde{b_i} = h^2 f_i \hspace{0.5cm}  i=1,2,3,..,n $ \\
$\beta_1 = \gamma_1 = 0$ \\
for $i=1,2,..,n-1$ \\ \vspace{0.5cm} 
 \hspace{0.5cm} $ \beta_{i+1} = \frac{\tilde{b_i}-a_i \beta_i}{a_i \gamma_i + b_i} , \hspace{0.5cm} \gamma_{i+1}=\frac{-c_i}{a_i \gamma_i + b_i} $ \vspace{0.2cm}  \\
 $v_{n+1} = 0$ \\
for $j=n+1, n,..,1$ \\ \vspace{0.5cm} 
 \hspace{0.5cm} $v_{i-1} = \beta_i + \gamma_i v_i$

\end{tcolorbox} 


This is often refered to as \emph{Thomas Algorithm}, an algorithm for solving tridiagonal systems of linear equations.




\end{document}\grid
