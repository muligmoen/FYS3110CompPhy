\documentclass[11pt,a4paper,english]{article}
\usepackage[english]{babel} % Using babel for norwegian hyphenation
\usepackage{lmodern} % Changing the font
\usepackage[utf8]{inputenc}
\usepackage[T1]{fontenc}

%\usepackage[moderate]{savetrees} % [subtle/moderate/extreme] really compact writing

\usepackage[parfill]{parskip} % Removes indents
\usepackage{amsmath} % Environment, symbols etc...
\usepackage{amssymb}
%\usepackage{wasysym} % Astrological symbols
\usepackage{graphicx} % For pictures etc...
\usepackage{enumitem} % Points/lists
\usepackage{physics} % Typesetting of mathematical physics examples: \bra{}, \ket{}, expval{}

% To include code(-snippets) with æøå
%\usepackage{listings}
%\lstset{
%language=python,
%showspaces=false,
%showstringspaces=false,
%frame=l,
%literate=%
%{æ}{{\ae}}1
%{ø}{{\o}}1
%{å}{{\aa}}1
%{Æ}{{\AE}}1
%{Ø}{{\O}}1
%{Å}{{\AA}}1
%}

\tolerance = 5000 % Bedre tekst
\hbadness = \tolerance
\pretolerance = 2000

\newcommand{\conj}[1]{#1^*}
\newcommand{\ve}[1]{\mathbf{#1}} % Vektorer i bold
\let\oldhat\hat
\renewcommand{\hat}[1]{\mathbf{\oldhat{#1}}}
\newcommand{\trans}[1]{#1^\top}
\newcommand{\herm}[1]{#1^\dagger}
%\renewcommand{\thefootnote}{\fnsymbol{footnote}} % Gir fotnote-symboler
\newcommand{\Real}{\mathbb{R}}


\newcommand{\spac}{\hspace{5mm}}

\title{FYS3150/FYS4150 \\Project 1}
\author{Magnus Ulimoen\\Krister Stræte Karlsen}
\date{\today}

\begin{document}
\maketitle

\section{Motivation}

Solutions of the Poissons equation are a vital component to many aspects physics.


\section{a)}

%Moving $h^2$ over and setting
%\begin{gather*}
%A = \begin{pmatrix} 2 & -1 & \ldots & \ldots & 0\\
%		    -1 & 2 & 0 & \ldots & \ldots \\
%		    \end{pmatrix}
%\end{gather*}
\subsection*{Formulation}

Given the following ODE with boundary conditions, 

\begin{equation}
-u''(x) = f(x), \spac x \in (0,1), \spac u(0) = u(1) = 0.
\end{equation}

and the dicretized form following a symmetric Taylor expansion

\begin{equation}
-\frac{v_{i+1}+v_{i-1}-2v_i}{h^2} = f_i  \spac \mathrm{for} \spac i=1,\dots, n, \spac v_0 = v_{n} = 0
\label{eq:discretized}
\end{equation}

we are going to show it can be written as system of linear equations of the form: 

\begin{equation}
   \ve{A}\ve{v} = \tilde{\ve{b}},
   \label{eq:Avb}
\end{equation}


\subsection*{Solution}
Multipling the discretized equation \eqref{eq:discretized} by $h^2$ we get:

\begin{gather*}
   -v_{i-1}+2v_{i}-v_{i+1}=h^2 f_i \hspace{0.5cm} \mathrm{for} \hspace{0.5cm} i=1,\dots, n 
\end{gather*}

Filling in for $i$ and choosing $\tilde{b_i} = h^2 f_i$ we obtain the following set of equations: 
\begin{align*}
	2v_{1}-v_{2}=\tilde{b_1} \\
	-v_{1}+2v_{2}-v_{3}=\tilde{b_2} \\
	\vdots \hspace{0.5cm} \\
	-v_{i-1}+2v_{i}-v_{i+1}=\tilde{b_i} \\
	\vdots \hspace{0.5cm}  \\ 
	-v_{n-1}+2v_{n}=\tilde{b_n} \\
\end{align*}
Now one can easily see that this system of linear equations can written on the form of \eqref{eq:Avb},
where 
\begin{gather*}
    \ve{A} = \begin{pmatrix}%{cccccc}
                           2& -1& 0 &\dots   & \dots &0 \\
                           -1 & 2 & -1 &0 &\dots &\dots \\
                           0&-1 &2 & -1 & 0 & \dots \\
                           & \dots   & \dots &\dots   &\dots & \dots \\
                           0&\dots   &  &-1 &2& -1 \\
                           0&\dots    &  & 0  &-1 & 2 \\
                      \end{pmatrix},\spac \ve{v} = \begin{pmatrix}
                           v_1\\
                           v_2\\
                           \dots \\
                          \dots  \\
                          \dots \\
                           v_n\\
                      \end{pmatrix},
  \spac \tilde{\ve{b}} = \begin{pmatrix}
                           \tilde{b}_1\\
                           \tilde{b}_2\\
                           \dots \\
                           \dots \\
                          \dots \\
                           \tilde{b}_n\\
                      \end{pmatrix}.
\end{gather*}





\end{document}